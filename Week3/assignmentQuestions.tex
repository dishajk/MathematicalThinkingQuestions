% \documentclass[12pt]{exam}
\documentclass[12pt, answers, a4paper]{exam}
\usepackage[margin=3cm]{geometry}
\usepackage{hyperref}
\usepackage{amssymb}
\usepackage{amsmath}
\usepackage{xparse}
\usepackage{physics}
\usepackage{tikz}
\usepackage{color}
\usepackage{framed}
\renewcommand{\thechoice}{\alph{choice}}
\renewcommand{\solutiontitle}{\noindent\textbf{Answer:}\enspace}

\SolutionEmphasis{\color{green}}
\title{\begin{framed}
    \textbf{Mathematical Thinking}\\Assignment Questions\\Week 3\\\normalsize{Total Marks: 20}
\end{framed}}
\date{}

\begin{document}
\maketitle

\setlength{\rightpointsmargin}{5pt}
\marginpointname{ \points}
\marksnotpoints
\pointpoints{mark}{marks}
\pointsinrightmargin

\unframedsolutions

\begin{questions}

\question  Let $b, n \in \mathbb{N}$ be such that $n \mid b$ and $n\mid (b+1).$ Find the value of $n$. 
\printanswers
\begin{solution}
    1
\end{solution}    

\question Let $k = \gcd(1092,5005).$ Use the Euclidean algorithm to compute $k$, then express $k$ as a linear combination of 1092 and 5005.
\printanswers
\begin{solution}
    $k =91$, $(23)1092 + (-5)(5005) = 91$
\end{solution}

\question Suppose $\gcd(a+b,a-b) = k$, where $a,b,k \in \mathbb{N}$. Which of the following options is (are) correct?
\begin{choices}
    \choice $\gcd(2a,2b) = k$
    \choice $\gcd(2a,a-b) = k$
    \choice $\gcd(a,b) = k$
    \choice $\gcd(a+b,2b) = k$
\end{choices}
\begin{solution}
    (b), (d)
\end{solution}

\question Every odd integer is of the form \fillin.
\begin{choices}
    \choice either $1 + (n-1)^2$ or $2+ (2n-1)^2, n \in \mathbb{N}$ 
    \choice either $6n-3$ or $6n-5, n \in \mathbb{N}$
    \choice $3n-2, n \in \mathbb{N}$
    \choice either $4n-1$ or $4n-3, n \in \mathbb{N}$
\end{choices}
\begin{solution}
    (d)
\end{solution}

\question[2] Suppose $\gcd(a,b) = 1$. If $c\mid a$ and $d\mid b$, then prove that $\gcd(c,d) = 1$.

\question[2]  If $a\mid bc$ and $\gcd(a,b) = 1$, then prove that $a\mid c$.
\noprintanswers
\begin{solution}
Since $\gcd(a, b) = 1$ we can write $1 = ax + by$. Therefore $c = acx + bcy$. But we know $a\mid acx$ and $a\mid bcy$, so $a\mid c$.    
\end{solution}





\question Using the induction method  prove that for any $n \in \mathbb{N}$, $2\mid n(n+1)$ and $3 \mid n(n+1)(n+2)$.

\question Consider two integers $x$ and $y$ such that $6x+12y=3$. Prove or disprove that such integers exist or not. If it exists then write some values of $x$ and $y$.
\end{questions}

\section*{graded}

\begin{questions}
\question[2] Find the remainder when $2^{20} + 3^{30} + 4^{40} + 5^{50}$ is divided by 7.
\begin{solution}
    6
\end{solution}
\question[2]  Let $a,b \in \mathbb{N}$. Which of the following options is (are) always true?
\begin{choices}
    \choice If $a^3 \mid b^3$, then $a \mid b$.
    \choice If $a^a \mid b^b$, then $a \mid b$.
    \choice If $a^b \mid b^a$, then $a \mid b$.
    \choice If $a^2 \mid 2b^2$, then $a \mid b$.
\end{choices}
\begin{solution}
    (a),(b),(d)
\end{solution}

\question[1] Let $m,n \in \mathbb{N}$ and $\gcd(m,n) = 1$. Which of the following options is (are) always true?
\begin{choices}
    \choice $\exists\ x, y \in \mathbb{Z}$ such that $mx - ny =1$.
    \choice $\exists\ x, y \in \mathbb{Z}$ such that $mx + ny = mn$.
    \choice $\gcd(xm, xn) = 1 \ \forall\ x \in \mathbb{Z}$. 
    \choice $\gcd(xm, yn) = 1\ \forall\ x,y \in \mathbb{Z}$ and $\gcd(x,y) =1$.
\end{choices}
\begin{solution}
    (a), (b)
\end{solution}

\question[2] Which of the following equations have solutions $a, b \in \mathbb{Z}$?
\begin{choices}

    \choice $12a + 20b = 42$
    \choice $152a + 102b = 3$
    \choice $23a + 11b = 120$
    \choice $21a + 91b = 50$
\end{choices}
\begin{solution}
    (c)
\end{solution}

\question[3] Consider a  number $a=a_1~a_2~a_3\hdots a_n$, where $a_1, a_2, a_3, \hdots , a_n$ are the digits for integers $a$. If $a$ is divisible by $8$, then prove that the number which is formed by the last 3 digits ($a_{n-2}~a_{n-1}~a_n$) of $a$ is divisible by 8.
\question[2] Using the division algorithm show that if $n \mid m$ and $n \mid k$, then $n^2\mid mk$.

\question[4] Prove that $\gcd(\gcd(m, n), k)=\gcd(m, \gcd(n, k))$.




\question[4] Let $r$ be the remainder obtained when $m$ is divided by $n$. Let  $g=\gcd(m, n)$ and $g'=\gcd(n, r)$. Prove that $g'\mid g$. 








\end{questions}
\end{document}