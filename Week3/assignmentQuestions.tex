% \documentclass[12pt]{exam}
\documentclass[12pt, answers, a4paper]{exam}
\usepackage[margin=3cm]{geometry}
\usepackage{hyperref}
\usepackage{amssymb}
\usepackage{amsmath}
\usepackage{xparse}
\usepackage{physics}
\usepackage{tikz}
\usepackage{color}
\usepackage{framed}
\usepackage{layout}
\renewcommand{\thechoice}{\alph{choice}}
\renewcommand{\solutiontitle}{\noindent\textbf{Answer:}\enspace}

\SolutionEmphasis{\color{teal}}
\title{\begin{framed}
    \textbf{Mathematical Thinking}\\Questions\\Week 3
\end{framed}}
\date{}

\begin{document}
\maketitle
\setlength{\rightpointsmargin}{5pt}
\marginpointname{ \points}
\pointsinrightmargin
\unframedsolutions

\begin{questions}
\question[1] Find the remainder when $2^{20} + 3^{30} + 4^{40} + 5^{50}$ is divided by 7.
\begin{solution}
    6
\end{solution}
\question[2] Consider a sequence $\{a_n\}$ such that $a_1 = 4$ and $a_{n+1} = 4^{a_n}$, for all $n > 1$. Find the remainder when $a_{100}$ is divided by 7.
\begin{solution}
    4
\end{solution}
\question[1]  Let $b, n \in \mathbb{N}$ be such that $n \mid b$ and $n\mid (b+1).$ Then prove that $n=1$
\question[2]  If $a\mid bc$ and $\gcd(a,b) = 1$, then prove that $a\mid c$.
\noprintanswers
\begin{solution}
Since $\gcd(a, b) = 1$ we can write $1 = ax + by$. Therefore $c = acx + bcy$. But we know $a\mid acx$ and $a\mid bcy$, so $a\mid c$.    
\end{solution}
\question[2] Suppose $\gcd(a,b) = 1$. If $c\mid a$ and $d\mid b$, then prove that $\gcd(c,d) = 1$.
\question[3] Let $k = \gcd(1092,5005).$ Use the Euclidean algorithm to compute $k$, then express $k$ as a linear combination of 1092 and 5005.
\printanswers
\begin{solution}
    $k =91$, $(23)1092 + (-5)(5005) = 91$
\end{solution}
\question Consider two integers $x$ and $y$ such that $6x+12y=3$. Prove or disprove that such integers exist or not. If it exists then  write some values of 
\end{questions}
\end{document}