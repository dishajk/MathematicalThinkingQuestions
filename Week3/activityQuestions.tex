\documentclass[12pt]{exam}
% \documentclass[12pt, answers]{exam}
\usepackage{hyperref}
\usepackage{amssymb}
\usepackage{amsmath}
\usepackage{xparse}
\usepackage{physics}
\usepackage{tikz}
\usepackage{color}
\title{Mathematical Thinking - Week 3}

\begin{document}
\maketitle
\tableofcontents
\section{Currency Game}
\begin{questions}
\question Can you find 3 others ways of paying Rs 1, using just 17 and 10 Rupee coins?
\question If the shopkeeper has no change, then how can we pay Rs. 144 using only 17 and 10 rupee coins? Can we pay Rs 143?
\end{questions}

\section{Divisibility}
\begin{questions}
\question Try your hand at proving the following:
\begin{parts}
\part If $a$ divides $b$, then $2a$ divides $6b$.
\part If $a$ divides $b$ and $c$ divides $d$, then $ac$ divides $bd$.
\part If $a$ divides $b$, then $a^2$ divides $b^2$
\part If $a^2$ divides $b^2$, then do you think $a$ must divide $b$? 
\end{parts}
\end{questions}

\section{Greatest Common Divisor}
\begin{questions}
\question If they exist, find all the possible solutions $m, n \in \mathbb{Z}$ for which,
\begin{parts}
    \part $89m + 97n = 150$
    \part $49m + 56n = 22$
    \part $315m + 189n = 42$
\end{parts}
\question Indian coins come as $1, 2, 5, 10$, and $20$ Rupee(s) coins. If you want to demonetise three of the coins and keep only coins of two values in circulation without affecting transaction, which pair(s) of coins would you choose and why? 
\end{questions}

\section{The Euclidean Algorithm}
\begin{questions}
    \question For $a, b \in \mathbb{Z}$, show that $\gcd(a, b) = \gcd(a - kb, b)$ for all $k \in \mathbb{N}$ using the principle of mathematical induction.
    \question Try your hand at computing the gcd of 900 and 55 using the Euclidean algorithm.
    \question If you double both numbers (i.e. take 1800 and 110), then what do you observe in the steps of the algorithm? How does the final gcd change?
\end{questions}
\section{Proof of the Euclidean Algorithm}
\begin{questions}
    \question The Euclidean algorithm eventually stops, as we showed. Can you make your own algorithm that takes two numbers as input and does something to them at each step, but with the property that the algorithm never stops.
    \question Can you modify your algorithm so that it stops for some values of input, but goes on forever for other choices?
\end{questions}

\section{Test for Divisibility}
\begin{questions}
        \question Construct the test for divisibility by 13. Hint: Use the method used for the construction of test for divisibility by 7.
\end{questions}
\end{document}