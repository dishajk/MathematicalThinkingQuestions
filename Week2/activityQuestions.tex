\documentclass[12pt]{exam}
% \documentclass[12pt, answers]{exam}
\usepackage{hyperref}
\usepackage{amssymb}
\usepackage{amsmath}
\usepackage{xparse}
\usepackage{physics}
\usepackage{tikz}
\usepackage{color}
\title{Mathematical Thinking - Week 2 \\ 
Activity Questions}

\begin{document}
\maketitle
\tableofcontents
\section{A Trip to Cantorsville}
\begin{questions}
    \question What are other ways in which the manager of the Hilbert Hotel in Cantorsville could have accommodated the people coming from the infinitely many Hilbert Hotels?

    % \question Instead of moving people from the $i$th Hilbert Hotel to rooms numbered odd multiples of $2^{i-1}$, as done by the managaer of the Hilbert Hotel in Cantorsville, what if,
    % \begin{parts}
    %     \part people from the first hotel were moved to room number $2^i$, where $i$ is their initial room number?
    %     \part people from the second hotel were moved to room number $3^i$, people from the third hotel to room number $5^i$ and so on.
    %     \part people from $j$th hotel to room number $p^i$ where $p$ is the $j$th prime number and $i$ is their original room number? 
    % \end{parts}
    \question What are other ways in which the manager of the Hilbert Hotel in Cantorsville could have accomodated the people coming from the infinitely many Hilbert Hotels if it is alright to leave some of the rooms empty?\\
    Hint: There are infinitely many prime numbers.
\end{questions}
\section{Cantor's Diagonalization Argument}
\begin{questions}
\question Each real number $r \in [0,1)$ can be denoted as $r_i = 0.d_{i1}d_{i2}d_{i3}d_{i4}\ldots$ for $i \in \mathbb{N}$, e.g. $\flatfrac{1}{2}$ can be written as $0.5000\ldots$ where $d_{i1} = 5$, $d_{i2} = 0$, $d_{i3} = 0$, and so on, for some $i \in \mathbb{N}$. Can you construct a bijection $f : \mathbb{N} \to [0,1)$? If not, use Cantor's Diagonalization Argument to show that such a function would be surjective.
\end{questions}
\section{Towards the Real Numbers}
\begin{questions}
    \question Show that there does not exist any rational number $x$ for which $x^2 = 3$.
    \question Verify each of the ten field axioms for rationals using the algebra of integers.
    \question Show that addition preserves order for rational numbers. If $a \geq b$, then $a + c \geq b + c$ for any $a, b, c \in \mathbb{Q}$.

    % positive addition preserves order
    % what about negative 
\end{questions}
\section{Ordered Field}
\begin{questions}
    \question Provide examples that illustrate each of the three order axioms for rational numbers.
    \question If $a > 0$ and $b < 0$. Show that $ab < 0$.
\end{questions}
\section{Completeness Axiom}
\begin{questions}
    \question In the lecture we find that $\mathbb{Q}$ has \textit{holes}. Give more examples of sets of rational numbers that does not have the least upper bound.
\end{questions}
\section{The Least Upper Bound Property}
\begin{questions}
    \question Assume that Amri runs the first half of the marathon in one hour and that his average speed each hour is half of his average speed in the previous hour. Work out how much of the marathon he would run in 6 hours? Will he ever completeness the marathon?  
    \question Using induction show that $2^n > n$ for every natural number $n$. Conclude that $2^{-n}<\flatfrac{1}{n}$ for every natural number $n$.
    % \question 
    % Last year, Disha and Viswanath both participated in the marathon. Unfortunately, Viswanath was delayed by his cab driver and arrived at the starting line late. Meanwhile, Disha had already run one kilometer. However, Viswanath was unconcerned as he is a faster runner than Disha. When he reached the one kilometer mark, he noticed that Disha was now 500 meters ahead of him. When he covered that 500m he noticed that Disha was now 250 meter ahead of him. Despite Viswanath’s efforts, he was unable to catch up to Disha. Is it possible for Viswanath to have overtaken Disha? If yes, what should have been the minimum length of the track?
\end{questions}
\section{Mathematical Logic and Statements}
\begin{questions}
    \question Show that for $a, b \in \mathbb{R}$, if $\abs{a - b} < \epsilon$ for all $\epsilon > 0$, then $a = b$. 
\end{questions}
% \begin{solution}
% \end{solution}
\end{document}