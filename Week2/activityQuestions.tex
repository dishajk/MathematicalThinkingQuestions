\documentclass[12pt]{exam}
% \documentclass[12pt, answers]{exam}
\usepackage{hyperref}
\usepackage{amssymb}
\usepackage{amsmath}
\usepackage{tikz}
\usepackage{color}
\title{Mathematical Thinking - Week 2}

\begin{document}
\maketitle
\tableofcontents
\section{A Trip to Cantorsville}
\begin{questions}
    \question What are other ways in which the managaer of the Hilbert Hotel in Cantorsville could have accomodated the people coming from the infinitely many Hilbert Hotels?

    % \question Instead of moving people from the $i$th Hilbert Hotel to rooms numbered odd multiples of $2^{i-1}$, as done by the managaer of the Hilbert Hotel in Cantorsville, what if,
    % \begin{parts}
    %     \part people from the first hotel were moved to room number $2^i$, where $i$ is their initial room number?
    %     \part people from the second hotel were moved to room number $3^i$, people from the third hotel to room number $5^i$ and so on.
    %     \part people from $j$th hotel to room number $p^i$ where $p$ is the $j$th prime number and $i$ is their original room number? 
    % \end{parts}
    \question What are other ways in which the managaer of the Hilbert Hotel in Cantorsville could have accomodated the people coming from the infinitely many Hilbert Hotels if it is alright to leave some of the rooms empty?\\
    Hint: There are infinitely many prime numbers.
\end{questions}
\section{Cantor's Diagonalization Argument}
\begin{questions}
\question Each real number $r \in [0,1)$ can be denoted as $r_i = 0.d_{i1}d_{i2}d_{i3}d_{i4}\ldots$ for $i \in \mathbb{N}$. Can you construct a bijection $f : \mathbb{N} \to [0,1)$? If not, use Cantor's Diagonalization Argument to show that such a function would be surjective.
% \question The Halting Problem is the problem of determining, from a description of an arbitrary computer program and an input, whether the program will finish running, or continue to run forever\footnote{Wikipedia contributors. "Halting problem." Wikipedia, The Free Encyclopedia. Wikipedia, The Free Encyclopedia, 24 Sep. 2023. Web. 26 Sep. 2023.}.\\
% Suppose we have an algorith $H$ that solves the Halting Problem. For any program $P$ and its input $x$,
% \begin{equation*}
%     H(P, x) = \left\{\begin{array}{l}
%         1\text{ if } P\text{ halts on input }x\\
%         0\text{ if } \text{ otherwise }x
%     \end{array}
%         \right.
% \end{equation*}
\end{questions}
\section{Towards the Real Numbers}
\begin{questions}
    \question Show that there does not exist any rational number $x$ for which $x^2 = 3$.\\
    \textcolor{red}{Is it too early to ask this question?}
    \question Verify each of the ten field axioms for rationals using the algebra of integers.
\end{questions}
\section{Ordered Field}
\begin{questions}
    \question
\end{questions}
\begin{solution}
\end{solution}
\end{document}