% \documentclass[12pt]{exam}
\documentclass[12pt, answers, a4paper]{exam}
\usepackage[margin=3cm]{geometry}
\usepackage{hyperref}
\usepackage{amssymb}
\usepackage{amsmath}
\usepackage{xparse}
\usepackage{physics}
\usepackage{tikz}
\usepackage{color}
\usepackage{framed}
\renewcommand{\thechoice}{\alph{choice}}
\renewcommand{\solutiontitle}{\noindent\textbf{Answer:}\enspace}

% \SolutionEmphasis{\color{green}}
\title{\begin{framed}
    \textbf{Mathematical Thinking}\\Activity Questions\\Week 6
\end{framed}}
\date{}

\begin{document}
\maketitle

\setlength{\rightpointsmargin}{5pt}
\marginpointname{ \points}
\marksnotpoints
\pointpoints{mark}{marks}
\pointsinrightmargin

\unframedsolutions
\section{Why Study Sequences?}
\begin{questions}
\question For the sequence $a_n$ constructed using the continued fraction representation of $\sqrt{2}$, show that
\begin{equation*}
    a_{n+1} = 1 + \frac{1}{1+a_n}
\end{equation*}
\question 
\begin{equation*}
    \text{Let } a_n = \left( 1 + \frac{1}{n} \right)^n.
\end{equation*}
Show that the $k$th term in the binomial expansion of $a_n$ is less than that of $a_{n+1}$.
\end{questions}
  \section{Binomial Coefficients}
\begin{questions}
    \question We want to form of committee of $k$ members from a group of $n$ people. One member from each committe is its chairperson. What are the number of ways in which the chairperson can be selected?
    \question There is a group of $n$ people from which one is selected to be the chairperson. The chairperson gets to pick the $k-1$ members from the group of $n-1$ people to form a committee of $k$ members.  
    \question From a group of 10 candidates, 5 committees of 5 members each are formed. In how many ways can we select a chairperson for each of these committees from its members?
    \question In how many ways
\question Evaluate $10\choose{5}$.
\begin{parts}
    \part How many ways can a committee of 5 people be selected from a group of 10 candidates?
    \part If you flip a coin 10 times, what is the probability of getting exactly 5 heads?
\end{parts}

\end{questions}
\section{Binomial Theorem}
\begin{questions}
    
\end{questions}
\end{document}