\documentclass[12pt]{amsart}
\usepackage{amsmath}
\usepackage{xparse}
\usepackage{physics}
\usepackage[none]{hyphenat}
\title{Activity 1.5}
\date{\today}
\begin{document}
\maketitle
\part*{Bijection and Cardinality}
\begin{enumerate}
\item What are the terms in the set $S_3 - S_2$?
\item The function $f: \mathbb{R} \to \mathbb{R}$ defined as $f(x) = x^3 - x$ is not injective. In what ways can the domain be restricted to make the function injective?
\item Consider the function $g: \mathbb{N} \to \mathbb{Z}$ defined as 
\begin{equation*}
g(n) = \left\{
    \begin{array}{ll}
        -\flatfrac{n}{2} & \text{if } n \text{ is even}\\
        \flatfrac{(n+1)}{2} & \text{if } n \text{ is odd}
    \end{array} 
    \right.
\end{equation*}
Is $g$ an injection, a surjection, or both - a bijection? If $g$ is a bijection, what does that say about the cardinality of $\mathbb{N}$ and $\mathbb{Z}$. Note that $\mathbb{N} \subset \mathbb{Z}$.
\item Try to construct a bijective function from $\mathbb{Z} \to \mathbb{N}$.
\end{enumerate}  
\end{document}